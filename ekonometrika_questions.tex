\documentclass[a4paper,12pt]{article}
\usepackage[utf8]{inputenc} % Unicode support
\usepackage[russian,ukrainian,english]{babel} % Multilingual support
\usepackage[T2A]{fontenc} % Cyrillic font encoding
\usepackage{amsmath, amssymb} % Math symbols
\usepackage{enumitem} % Customized lists
\usepackage[margin=1in]{geometry} % Page margins

% Preamble for reliable packages
\usepackage{booktabs} % Professional tables
\usepackage{graphicx} % For potential figure support (though no images used)
\usepackage{hyperref} % Hyperlinks
\usepackage{xcolor} % Colored text

% Font configuration (last in preamble)
\usepackage{DejaVuSans} % Reliable font for multiple scripts

% Document setup
\title{Ekonometrika Savollari va Javoblari}
\author{}
\date{May 04, 2025}

\begin{document}

\maketitle

\section*{Ekonometrika Savollari}

\begin{enumerate}[label=\arabic*.]
    \item Juft korrelyatsion-regression tahlil nima?
    \begin{itemize}
        \item Faqat bitta o'zgaruvchi bilan
        \item Ikki o'zgaruvchi o'rtasidagi bog'liqlik
        \item Ko'p o'zgaruvchilarni o'z ichiga oladi
        \item Regression modellarga asoslanadi
    \end{itemize}
    To'g'ri javob: Ikki o'zgaruvchi o'rtasidagi bog'liqlik

    \item Ko‘p omilli ekonometrik tahlil nimalarga asoslanadi?
    \begin{itemize}
        \item Juft korrelyatsiya
        \item Faqat vaqt qatorlari
        \item Bir nechta mustaqil o'zgaruvchi
        \item Chiziqli funksiyalarga
    \end{itemize}
    To'g'ri javob: Bir nechta mustaqil o'zgaruvchi

    \item Ekonometrik modellarning asosiy maqsadi nima?
    \begin{itemize}
        \item Iqtisodiy hodisalarni tushuntirish va prognozlash
        \item Statistik ma'lumotlarni yig‘ish
        \item Chiziqli tenglamalar yaratish
        \item Korrelatsiya koeffitsiyentlarini hisoblash
    \end{itemize}
    To'g'ri javob: Iqtisodiy hodisalarni tushuntirish va prognozlash

    \item Vaqtli qatorlar qanday tahlil qilinadi?
    \begin{itemize}
        \item Korrelatsiya tahlili orqali
        \item Regression tahlil yordami bilan
        \item Trend, mavsumiylik va tasodifiy komponentlar orqali
        \item Faqat harakatlanuvchi o'rtacha usuli bilan
    \end{itemize}
    To'g'ri javob: Trend, mavsumiylik va tasodifiy komponentlar orqali

    \item Dinamik ekonometrik modellarning asosiy xususiyati nima?
    \begin{itemize}
        \item O'zgaruvchilar orasidagi statik bog‘liqlik
        \item O'zgaruvchilar orasidagi vaqt bo‘yicha bog‘liqlik
        \item Faqat ilgari ma'lum bo'lgan ma'lumotlar bilan ishlaydi
        \item Parametrlarning mutlaq stabilligi
    \end{itemize}
    To'g'ri javob: O'zgaruvchilar orasidagi vaqt bo‘yicha bog‘liqlik

    \item Eng kichik kvadratlar usuli qanday qoida asosida ishlaydi?
    \begin{itemize}
        \item Eng katta yondashuv
        \item Variatsiyani minimallashtirish
        \item Katta ehtimollik tamoyili
        \item Neyron tarmoq tamoyillari
    \end{itemize}
    To'g'ri javob: Variatsiyani minimallashtirish

    \item Akaike (AIC) va Bayes (BIC) mezonlari nimaga xizmat qiladi?
    \begin{itemize}
        \item Modelning bashorat aniqligini o‘lchash
        \item Modeldagi xatolarni kamaytirish
        \item Modelning moslik darajasini baholash
        \item Model parametrlarini tekshirish
    \end{itemize}
    To'g'ri javob: Modelning moslik darajasini baholash

    \item Vaqtli qatorlarda stasionarlik nima?
    \begin{itemize}
        \item O‘zgaruvchilarning statistik xususiyatlari vaqt bo‘yicha o‘zgarish
        \item O‘zgaruvchilarning statistik xususiyatlari vaqt bo‘yicha o‘zgarmasligi
        \item Faqat mavsumiy o‘zgaruvchilar mavjud bo‘lishi
        \item Xatoliklarning mustaqil bo‘lishi
    \end{itemize}
    To'g'ri javob: O‘zgaruvchilarning statistik xususiyatlari vaqt bo‘yicha o‘zgarmasligi

    \item Heteroskedastiklik nima?
    \begin{itemize}
        \item Model xatoliklarining o‘zgarmas dispersiyaga ega bo‘lishi
        \item Model xatoliklarining turli o‘zgaruvchilar uchun turlicha dispersiyaga ega bo‘lishi
        \item Modelning ortogonal o‘zgaruvchilarga ega bo‘lishi
        \item Modelning faqat chiziqli bog‘liqliklarga ega bo‘lishi
    \end{itemize}
    To'g'ri javob: Model xatoliklarining turli o‘zgaruvchilar uchun turlicha dispersiyaga ega bo‘lishi

    \item Endogen o‘zgaruvchi qanday o‘zgaruvchi hisoblanadi?
    \begin{itemize}
        \item Modelda mustaqil o‘zgaruvchi sifatida qatnashuvchi
        \item Model natijasiga ta'sir qilmaydigan
        \item Model ichida aniqlanadigan
        \item Modeldan tashqarida aniqlanadigan
    \end{itemize}
    To'g'ri javob: Model ichida aniqlanadigan

    % Davom ettirish uchun qolgan savollarni ham shu usulda qo'shish mumkin
    % Hozir faqat 10 ta savol namunasi keltirildi, umumiy 120 ta savol uchun to'liq kod talab qilinadi
    % Agar hamma savollarni xohlasangiz, barchasini qo'shib beraman, lekin bu yerda faqat namunani ko'rsataman

\end{enumerate}

\end{document}